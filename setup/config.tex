% !TeX root = ../main.tex

% --------------------------------------------------
% 論文內容選項
% --------------------------------------------------

% 論文是否為初稿 Y: 是 N: 否
% 若選擇是則不加入前頁的 PDF 文件,如審定書、授權書
% \newcommand{\thesisIsDraft}{Y}
\newcommand{\thesisIsDraft}{N}

% 學位論文類別 BA: 學士 MA: 碩士 PHD: 博士
% \newcommand{\thesisDegreeType}{BA}
\newcommand{\thesisDegreeType}{MA}
% \newcommand{\thesisDegreeType}{PHD}

% 是否有共同指導教授 Y: 有 N: 無
\newcommand{\thesisHasCoAdvisor}{Y}
% \newcommand{\thesisHasCoAdvisor}{N}

% 是否使用浮水印 Y: 是 N: 否
\newcommand{\thesisUseWatermark}{Y}
% \newcommand{\thesisUseWatermark}{N}

% 中文字體 預設標楷體 BiauKai
% 其他選項見 fonts/zhtw 欲修改使用字體全稱 
\newcommand{\thesisFontsZHTW}{BiauKai} % 標楷體
% \newcommand{\thesisFontsZHTW}{edukai-4.0} % 教育部楷體
% \newcommand{\thesisFontsZHTW}{edusong-big5} % 教育部宋體

% 是否有符號說明表 Y: 是 N: 否
\newcommand{\thesisHasDenotation}{Y}
% \newcommand{\thesisHasDenotation}{N}

% 是否區分中英文獻 Y: 是 N: 否
% 若要使用此選項,則 references.bib 中的中文文獻需要確實設定 keyword={chinese}
\newcommand{\thesisSplitRefByLang}{Y}
% \newcommand{\thesisSplitRefByLang}{N}

% 是否顯示邊界線 Y: 顯示 N: 不顯示
% 顯示邊界線可以在撰寫時幫助判斷內容顯示範圍
\newcommand{\thesisShowBoardLine}{Y}
% \newcommand{\thesisShowBoardLine}{N}

% --------------------------------------------------
% 論文基本資訊
% --------------------------------------------------

% 研究生資訊
\newcommand{\thesisAuthorNameZHTW}{阿烏拉}
\newcommand{\thesisAuthorNameEN}{A WU LA}

% 指導教授
\newcommand{\thesisAdvisorNameZHTW}{芙利蓮}
\newcommand{\thesisAdvisorNameEN}{Frieren}

% 共同指導教授(若有共指需要修改)
\newcommand{\thesisCoAdvisorNameZHTW}{費倫}
\newcommand{\thesisCoAdvisorNameEN}{Fern}

% 學校資訊
\newcommand{\thesisSchoolNameZHTW}{國立高雄科技大學}
\newcommand{\thesisSchoolNameEN}{National Kaohsiung University of Science and Technology}
\newcommand{\thesisSchoolLocationEN}{Kaohsiung, Taiwan, Republic of China}

% 學院(保留)
% \newcommand{\thesisCollegeNameZHTW}{工學院}
% \newcommand{\thesisCollegeNameEN}{College of Engineering}

% 系名(=主修)
\newcommand{\thesisDeptNameZHTW}{電子工程}
\newcommand{\thesisDeptNameEN}{Electronic Engineering}

% 班名
\newcommand{\thesisClassNameZHTW}{\thesisDeptNameZHTW 系碩士班}
\newcommand{\thesisClassNameEN}{Graduate Institute of \thesisDeptNameEN}

% 學士學位與授予學位(僅修正選擇學位即可)
\newcommand{\thesisDegreeBachelorZHTW}{學士}
\newcommand{\thesisDegreeBachelorEN}{Bachelor}
\newcommand{\thesisConferredDegreeBachelor}{\thesisDegreeBachelorEN~of Science}

% 碩士學位與授予學位(僅修正選擇學位即可)
\newcommand{\thesisDegreeMasterZHTW}{碩士}
\newcommand{\thesisDegreeMasterEN}{Master}
\newcommand{\thesisConferredDegreeMaster}{\thesisDegreeMasterEN~of Science}

% 博士學位與授予學位(僅修正選擇學位即可)
\newcommand{\thesisDegreeDoctorZHTW}{博士}
\newcommand{\thesisDegreeDoctorEN}{Doctor}
\newcommand{\thesisConferredDegreeDoctor}{\thesisDegreeDoctorEN~of Science}

% 論文題目
\newcommand{\thesisTitleZHTW}{高雄科技大學 \LaTeX 論文樣板}
\newcommand{\thesisTitleEN}{NKUST \LaTeX~Thesis Template}

% 論文關鍵字
\newcommand{\thesisKeywordsZHTW}{{LaTeX},{高科},{論文},{模板}}
\newcommand{\thesisKeywordsEN}{{LaTeX},{NKUST},{Thesis},{Template}}

% 論文日期月份
\newcommand{\thesisYearAD}{2024} % 西元年數字
\newcommand{\thesisYearROC}{\the\numexpr\thesisYearAD-1911\relax} % 西元年 - 1911
\newcommand{\thesisMonth}{7} % 月份數字

\newcommand{\thesisDateROC}{中華民國\zhdigits{\thesisYearROC}年\zhdigits{\thesisMonth}月}
\newcommand{\thesisDateEN}{\thesisGetMonthName{\thesisMonth}~\thesisYearAD}

% --------------------------------------------------
% 語言變數
% --------------------------------------------------

% 繁體中文摘要
% 摘要中英兩邊都要有,所以用詞必須分開列出
\newcommand{\thesisAbstractTitleZHTW}{摘要}
\newcommand{\thesisKeywordsTagZHTW}{關鍵字:}
\newcommand{\thesisKeywordsSymbolZHTW}{、}

% 英文摘要
\newcommand{\thesisAbstractTitleEN}{ABSTRACT}
\newcommand{\thesisKeywordsTagEN}{Keywords:~}
\newcommand{\thesisKeywordsSymbolEN}{,~}

% ------ 繁體中文用詞(若使用中文解開以下變數註解) -----

% 繁體中文行高
\linespread{1.5}

% 繁體中文章節標題間隔
\newcommand{\thesisTitleGapSize}{0em}

% 繁體中文標題用詞
\newcommand{\thesisOralExamApprovalTitle}{口試委員審定書}
\newcommand{\thesisAcknowledgementTitle}{致謝}
\newcommand{\thesisTableOfContentsTitle}{目錄}
\newcommand{\thesisListOfFiguresTitle}{圖目錄}
\newcommand{\thesisListOfTablesTitle}{表目錄}
\newcommand{\thesisDenotationTitle}{符號說明}
\newcommand{\thesisBibliographyTitle}{參考文獻}
\newcommand{\thesisAppendixTitle}{附錄}

% 繁體中文圖表標題名
\newcommand{\thesisFigureName}{圖}
\newcommand{\thesisTableName}{表}

% 繁體中文章標題編號格式,參考 zhnumber counters
% 內文,參數:#1 => 章計數器值(chapter)
\newcommand{\thesisChapterTitleNumbering}[1]{\zhnum{#1}}
% 附錄,參數:#1 => 章計數器值(chapter)
\newcommand{\thesisAppendixChapterTitleNumbering}[1]{\zhnum{#1}}

% 繁體中文節標題編號格式
% 內文,參數:#1 => 章計數器值(chapter)、#2 => 節計數器值(section)
\newcommand{\thesisSectionTitleNumbering}[2]{\arabic{#1}.\arabic{#2}}
% 附錄,參數:#1 => 章計數器值(chapter)、#2 => 節計數器值(section)
\newcommand{\thesisAppendixSectionTitleNumbering}[2]{\arabic{#1}.\arabic{#2}}

% 繁體中文章標題顯示格式
% 內文
\newcommand{\thesisChapterTitleFormat}{\thechapter、}
% 目錄
\newcommand{\thesisChapterTitleTocFormat}{\thecontentslabel、}

% 繁體中文附錄章標題顯示格式
% 內文
\newcommand{\thesisAppendixTitleFormat}{\thesisAppendixTitle\thechapter、}
% 目錄
\newcommand{\thesisAppendixTitleTocFormat}{\thesisAppendixTitle\thecontentslabel、}

% 繁體中文節標題顯示格式(保留)

% 繁體中文附錄節標題顯示格式(保留)

% 繁體中文子節標題顯示格式(保留)

% 繁體中文附錄子節標題顯示格式(保留)


% ---------- 英文用詞(若使用英文解開以下變數註解) --------

% % 英文行高
% \linespread{1.5}

% % 英文章節標題間隔大小
% \newcommand{\thesisTitleGapSize}{1em}

% % 英文標題用詞
% \newcommand{\thesisOralExamApprovalTitle}{The Oral Examination Committee Approval}
% \newcommand{\thesisTableOfContentsTitle}{Contents}
% \newcommand{\thesisListOfFiguresTitle}{List of Figures}
% \newcommand{\thesisListOfTablesTitle}{List of Tables}
% \newcommand{\thesisDenotationTitle}{Denotation}
% \newcommand{\thesisAcknowledgementTitle}{Acknowledgements}
% \newcommand{\thesisBibliographyTitle}{References}
% \newcommand{\thesisAppendixTitle}{Appendix}

% % 英文圖表標題名
% \newcommand{\thesisFigureName}{Figure}
% \newcommand{\thesisTableName}{Table}

% % 英文章標題編號格式,參考 LaTeX 內建 counters
% % 內文,參數:#1 => 章計數器值(chapter)
% \newcommand{\thesisChapterTitleNumbering}[1]{\arabic{#1}}
% % 附錄,參數:#1 => 章計數器值(chapter)
% \newcommand{\thesisAppendixChapterTitleNumbering}[1]{\Alph{#1}}

% % 英文節標題編號格式,參考 LaTeX 內建 counters
% % 內文,參數:#1 => 章計數器值(chapter)、#2 => 節計數器值(section)
% \newcommand{\thesisSectionTitleNumbering}[2]{\arabic{#1}.\arabic{#2}}
% % 附錄,參數:#1 => 章計數器值(chapter)、#2 => 節計數器值(section)
% \newcommand{\thesisAppendixSectionTitleNumbering}[2]{\Alph{#1}.\arabic{#2}}

% % 英文章標題顯示格式
% % 內文
% \newcommand{\thesisChapterTitleFormat}{Chapter~\thechapter}
% % 目錄
% \newcommand{\thesisChapterTitleTocFormat}{Chapter~\thecontentslabel\hspace{\thesisTitleGapSize}}

% % 英文附錄章標題顯示格式
% % 內文
% \newcommand{\thesisAppendixTitleFormat}{\thesisAppendixTitle~\thechapter}
% % 目錄
% \newcommand{\thesisAppendixTitleTocFormat}{\thesisAppendixTitle~\thecontentslabel\hspace{\thesisTitleGapSize}}

% 英文節標題顯示格式(保留)

% 英文附錄節標題顯示格式(保留)

% 英文子節標題顯示格式(保留)

% 英文附錄子節標題顯示格式(保留)

% --------------------------------------------------
% 路徑變數(盡量勿動)
% --------------------------------------------------

% 審定書、授權書、推薦書
\newcommand{\thesisOralExamApprovalPDF}{./front/oral_exam_approval.pdf}
\newcommand{\thesisPowerOfAttorneyPDF}{./front/power_of_attorney.pdf}
\newcommand{\thesisAdvisorRecommendationPDF}{}

% 參考文獻
\newcommand{\thesisRefBIB}{./back/references.bib}

% 字體檔案
\newcommand{\thesisFontsPathZHTW}{./fonts/zhtw/}
\newcommand{\thesisFontsPathEN}{./fonts/en/}

% 圖片根目錄
\newcommand{\thesisFiguresPath}{./figures}

% 校徽與浮水印
\newcommand{\thesisSealImage}{./figures/seal/seal_golden.png}
\newcommand{\thesisWatermarkImage}{./figures/seal/seal_golden.png}

% --------------------------------------------------
% 排版變數
% --------------------------------------------------

% 繁體中文目錄排版單位設定(保留)


% 英文目錄排版單位設定(保留)
