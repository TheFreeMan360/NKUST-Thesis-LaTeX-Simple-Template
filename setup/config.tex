% !TeX root = ../main.tex

% --------------------------------------------------
% 論文內容選項
% --------------------------------------------------

% 論文是否為初稿 Y: 是 N: 否
% 若選擇是則不加入前頁的 PDF 文件,如審定書、授權書
% \newcommand{\thesisIsDraft}{Y}
\newcommand{\thesisIsDraft}{N}

% 學位論文類別 BA: 學士 MA: 碩士 PHD: 博士
% \newcommand{\thesisDegreeType}{BA}
\newcommand{\thesisDegreeType}{MA}
% \newcommand{\thesisDegreeType}{PHD}

% 是否有共同指導教授 Y: 有 N: 無
\newcommand{\thesisHasCoAdvisor}{Y}
% \newcommand{\thesisHasCoAdvisor}{N}

% 是否使用浮水印 Y: 是 N: 否
\newcommand{\thesisUseWatermark}{Y}
% \newcommand{\thesisUseWatermark}{N}

% 是否使用中文撰寫 Y: 是 N: 否
\newcommand{\thesisUseZHTW}{Y}
% \newcommand{\thesisUseZHTW}{N}

% 中文字體 預設標楷體 BiauKai
% 其他選項見 fonts/zhtw 欲修改使用字體全稱 
\newcommand{\thesisFontsZHTW}{BiauKai} % 標楷體
% \newcommand{\thesisFontsZHTW}{edukai-4.0} % 教育部楷體
% \newcommand{\thesisFontsZHTW}{edusong-big5} % 教育部宋體

% 是否有符號說明表 Y: 是 N: 否
\newcommand{\thesisHasDenotation}{Y}
% \newcommand{\thesisHasDenotation}{N}

% 是否顯示邊界線 Y: 顯示 N: 不顯示
% 顯示邊界線可以在撰寫時幫助判斷內容顯示範圍
\newcommand{\thesisShowBoardLine}{Y}
% \newcommand{\thesisShowBoardLine}{N}

% 是否區分中英文獻 Y: 是 N: 否
% 若要使用,references.bib 中文文獻需要確實設定 keyword={chinese} 
\newcommand{\thesisSplitBIBByLang}{Y}
% \newcommand{\thesisSplitBIBByLang}{N}

% --------------------------------------------------
% 論文基本資訊
% --------------------------------------------------

% 研究生資訊
\newcommand{\thesisAuthorNameZHTW}{阿烏拉}
\newcommand{\thesisAuthorNameEN}{A WU LA}

% 指導教授
\newcommand{\thesisAdvisorNameZHTW}{芙利蓮}
\newcommand{\thesisAdvisorNameEN}{Frieren}

% 共同指導教授(有共指需要修改)
\newcommand{\thesisCoAdvisorNameZHTW}{費倫}
\newcommand{\thesisCoAdvisorNameEN}{Fern}

% 學校資訊
\newcommand{\thesisSchoolNameZHTW}{國立高雄科技大學}
\newcommand{\thesisSchoolNameEN}{National Kaohsiung University of Science and Technology}
\newcommand{\thesisSchoolLocationEN}{Kaohsiung, Taiwan, Republic of China}

% 學院
% \newcommand{\thesisCollegeNameZHTW}{工學院}
% \newcommand{\thesisCollegeNameEN}{College of Engineering}

% 系名(=主修)
\newcommand{\thesisDeptNameZHTW}{電子工程}
\newcommand{\thesisDeptNameEN}{Electronic Engineering}

% 班名
\newcommand{\thesisClassNameZHTW}{\thesisDeptNameZHTW 系碩士班}
\newcommand{\thesisClassNameEN}{Graduate Institute of \thesisDeptNameEN}

% 學士學位與授予學位(僅修正選擇學位即可)
\newcommand{\thesisDegreeBachelorZHTW}{學士}
\newcommand{\thesisDegreeBachelorEN}{Bachelor}
\newcommand{\thesisConferredDegreeBachelor}{\thesisDegreeBachelorEN~of Science}

% 碩士學位與授予學位(僅修正選擇學位即可)
\newcommand{\thesisDegreeMasterZHTW}{碩士}
\newcommand{\thesisDegreeMasterEN}{Master}
\newcommand{\thesisConferredDegreeMaster}{\thesisDegreeMasterEN~of Science}

% 博士學位與授予學位(僅修正選擇學位即可)
\newcommand{\thesisDegreeDoctorZHTW}{博士}
\newcommand{\thesisDegreeDoctorEN}{Doctor}
\newcommand{\thesisConferredDegreeDoctor}{\thesisDegreeDoctorEN~of Science}

% 論文題目
\newcommand{\thesisTitleZHTW}{高雄科技大學 \LaTeX 論文樣板}
\newcommand{\thesisTitleEN}{NKUST \LaTeX~Thesis Template}

% 論文關鍵字
\newcommand{\thesisKeywordsZHTW}{{LaTeX},{高科},{論文},{模板}}
\newcommand{\thesisKeywordsEN}{{LaTeX},{NKUST},{Thesis},{Template}}

% 論文日期月份
\newcommand{\thesisYearAD}{2024} % 西元年數字
\newcommand{\thesisYearROC}{\the\numexpr\thesisYearAD-1911\relax} % 西元年 - 1911
\newcommand{\thesisMonth}{7} % 月份數字

\newcommand{\thesisDateROC}{中華民國\zhdigits{\thesisYearROC}年\zhdigits{\thesisMonth}月}
\newcommand{\thesisDateEN}{\thesisGetMonthName{\thesisMonth} \thesisYearAD}

% --------------------------------------------------
% 路徑變數
% --------------------------------------------------

% 審定書、授權書、推薦書
\newcommand{\thesisOralExamApprovalPDF}{./front/oral_exam_approval.pdf}
\newcommand{\thesisPowerOfAttorneyPDF}{./front/power_of_attorney.pdf}
\newcommand{\thesisAdvisorRecommendationPDF}{}

% 參考文獻
\newcommand{\thesisRefBIB}{./back/references.bib}

% 字體檔案
\newcommand{\thesisFontsPathZHTW}{./fonts/zhtw/}
\newcommand{\thesisFontsPathEN}{./fonts/en/}

% 圖片根目錄
\newcommand{\thesisFiguresPath}{./figures}

% 校徽與浮水印
\newcommand{\thesisSealImage}{./figures/seal/seal_golden.png}
\newcommand{\thesisWatermarkImage}{./figures/seal/seal_golden.png}

% --------------------------------------------------
% 語言變數
% --------------------------------------------------

% 繁體中文摘要(不管怎樣都會用到,不加入選項)
\newcommand{\thesisAbstractTitleZHTW}{摘要}
\newcommand{\thesisKeywordsTagZHTW}{關鍵字:}
\newcommand{\thesisKeywordsSymbolZHTW}{、}

% 英文摘要(不管怎樣都會用到,不加入選項)
\newcommand{\thesisAbstractTitleEN}{ABSTRACT}
\newcommand{\thesisKeywordsTagEN}{Keywords:~}
\newcommand{\thesisKeywordsSymbolEN}{,~}

% 繁體中文標題用詞
\newcommand{\thesisOralExamApprovalTitleZHTW}{口試委員審定書}
\newcommand{\thesisAcknowledgementTitleZHTW}{致謝}
\newcommand{\thesisTableOfContentsTitleZHTW}{目錄}
\newcommand{\thesisListOfFiguresTitleZHTW}{圖目錄}
\newcommand{\thesisListOfTablesTitleZHTW}{表目錄}
\newcommand{\thesisDenotationTitleZHTW}{符號說明}
\newcommand{\thesisBibliographyTitleZHTW}{參考文獻}
\newcommand{\thesisAppendixTitleZHTW}{附錄}

% 繁體中文圖表標題名
\newcommand{\thesisFigureNameZHTW}{圖}
\newcommand{\thesisTableNameZHTW}{表}

% 繁體中文章標題顯示格式
\newcommand{\thesisChapterTitleFormatZHTW}[1]{\zhnumber{#1}、}

% 繁體中文附錄章標題顯示格式
\newcommand{\thesisAppendixTitleFormatZHTW}[1]{\thesisAppendixTitleZHTW\zhnumber{#1}、}

% 繁體中文節標題顯示格式

% 繁體中文附錄節標題顯示格式

% 繁體中文子節標題顯示格式

% 繁體中文附錄子節標題顯示格式

% 英文標題用詞
\newcommand{\thesisOralExamApprovalTitleEN}{The Oral Examination Committee Approval}
\newcommand{\thesisTableOfContentsTitleEN}{Contents}
\newcommand{\thesisListOfFiguresTitleEN}{List of Figures}
\newcommand{\thesisListOfTablesTitleEN}{List of Tables}
\newcommand{\thesisDenotationTitleEN}{Denotation}
\newcommand{\thesisAcknowledgementTitleEN}{Acknowledgements}
\newcommand{\thesisBibliographyTitleEN}{References}
\newcommand{\thesisAppendixTitleEN}{Appendix}

% 英文圖表標題名
\newcommand{\thesisFigureNameEN}{Figure}
\newcommand{\thesisTableNameEN}{Table}

% 英文標題顯示格式
\newcommand{\thesisChapterTitleFormatEN}[1]{Chapter~{#1}~}

% 英文附錄章標題顯示格式
\newcommand{\thesisAppendixTitleFormatEN}[1]{\thesisAppendixTitleEN~{#1}~}

% 英文節標題顯示格式

% 英文附錄節標題顯示格式

% 英文子節標題顯示格式

% 英文附錄子節標題顯示格式


% --------------------------------------------------
% 排版變數
% --------------------------------------------------

% 繁體中文目錄排版單位設定(保留)


% 英文目錄排版單位設定(保留)
