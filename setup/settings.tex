% !TeX root = ../main.tex

% --------------------------------------------------
% 論文資訊設定
% --------------------------------------------------

% 學位類型設定
\ifthenelse{\equal{\thesisDegreeType}{BA}}{
    \newcommand{\thesisDegreeZHTW}{\thesisDegreeBachelorZHTW}
    \newcommand{\thesisDegreeEN}{\thesisDegreeBachelorEN}
    \newcommand{\thesisConferredDegree}{\thesisConferredDegreeBachelor}
}{}

\ifthenelse{\equal{\thesisDegreeType}{MA}}{
    \newcommand{\thesisDegreeZHTW}{\thesisDegreeMasterZHTW}
    \newcommand{\thesisDegreeEN}{\thesisDegreeMasterEN}
    \newcommand{\thesisConferredDegree}{\thesisConferredDegreeMaster}
}{}

\ifthenelse{\equal{\thesisDegreeType}{PHD}}{
    \newcommand{\thesisDegreeZHTW}{\thesisDegreeDoctorZHTW}
    \newcommand{\thesisDegreeEN}{\thesisDegreeDoctorEN}
    \newcommand{\thesisConferredDegree}{\thesisConferredDegreeDoctor}
}{}

% --------------------------------------------------
% 語言與字體設定
% --------------------------------------------------
% - \setmainfont{}          	襯線字體
% - \setsansfont{}          	非襯字體
% - \setmonofont{}          	等寬字體
% - \setCJKmainfont{}			中文襯線字體
% - \setCJKsansfont{}			中文非襯字體
% - \setCJKmonofont{}			中文等寬字體
% --------------------------------------------------

% zhnumber 初始設定 
\zhnumsetup{style={Traditional}} % 以繁中顯示

% 預設中文字體(CJK)設定
\xeCJKsetup{
    AutoFakeBold=true, % 沒有給定粗體時使用偽粗體
    EmboldenFactor=2.5, % 偽粗體的粗度
    AutoFakeSlant=true, % 沒有給定斜體時使用偽斜體
    SlantFactor=0.2, % 偽斜體的斜度
    PunctStyle=quanjiao, % 標點符號為全形
}

\setCJKmainfont[
    Path = \thesisFontsPathZHTW ,
    Extension = .ttf ,
]{ \thesisFontsZHTW }

% 預設英文字體設定
\setmainfont[
    Path = \thesisFontsPathEN ,
    Extension = .ttf ,
    BoldFont = *-Bold ,
    ItalicFont = *-Italic ,
    BoldItalicFont = *-BoldItalic,
]{Times New Roman}

% 撰寫語言選項
\ifthenelse{\equal{\thesisUseZHTW}{Y}}{
    % 繁體中文章節、標題、目錄等用詞
    \newcommand{\thesisOralExamApproval}{\thesisOralExamApprovalZHTW}
    \newcommand{\thesisTableOfContents}{\thesisTableOfContentsZHTW}
    \newcommand{\thesisListOfFigures}{\thesisListOfFiguresZHTW}
    \newcommand{\thesisListOfTables}{\thesisListOfTablesZHTW}
    \newcommand{\thesisDenotation}{\thesisDenotationZHTW}
    \newcommand{\thesisAcknowledgementTitle}{\thesisAcknowledgementTitleZHTW}

    \newcommand{\thesisChapterTitle}{\thesisChapterTitleZHTW}
    \newcommand{\thesisChapterTitleToc}{\thesisChapterTitleTocZHTW}

    \newcommand{\thesisBibliographyTitle}{\thesisBibliographyTitleZHTW}
    \newcommand{\thesisAppendixTitle}{\thesisAppendixTitleZHTW}

    \newcommand{\thesisFigureName}{\thesisFigureNameZHTW}
    \newcommand{\thesisTableName}{\thesisTableNameZHTW}

    % 中文目錄排版單位(保留)

}{
    % 英文章節、標題、目錄等用詞
    \newcommand{\thesisOralExamApproval}{\thesisOralExamApprovalEN}
    \newcommand{\thesisTableOfContents}{\thesisTableOfContentsEN}
    \newcommand{\thesisListOfFigures}{\thesisListOfFiguresEN}
    \newcommand{\thesisListOfTables}{\thesisListOfTablesEN}
    \newcommand{\thesisDenotation}{\thesisDenotationEN}
    \newcommand{\thesisAcknowledgementTitle}{\thesisAcknowledgementTitleEN}

    \newcommand{\thesisChapterTitle}{\thesisChapterTitleEN}
    \newcommand{\thesisChapterTitleToc}{\thesisChapterTitleTocEN}

    \newcommand{\thesisBibliographyTitle}{\thesisBibliographyTitleEN}
    \newcommand{\thesisAppendixTitle}{\thesisAppendixTitleEN}

    \newcommand{\thesisFigureName}{\thesisFigureNameEN}
    \newcommand{\thesisTableName}{\thesisTableNameEN}

    % 英文目錄排版單位(保留)

}

% 取得月份英文
\newcommand{\thesisGetMonthName}[1]{%
    \ifcase#1
    \or January% 1
    \or February% 2
    \or March% 3
    \or April% 4
    \or May% 5
    \or June% 6
    \or July% 7
    \or August% 8
    \or September% 9
    \or October% 10
    \or November% 11
    \or December% 12
    \else Invalid month% 如果超出範圍,則返回 "Invalid month"
    \fi
}

% --------------------------------------------------
% 版面設定
% --------------------------------------------------

% geometry 初始版面設定
\geometry{
    paper=a4paper,
    includehead, % 將 headheight headsep 計入 top
    headheight=2cm,
    headsep=1cm,
    top=1cm,
    includefoot,
    bottom=3cm,
    left=3cm,
    right=3cm,
}

% 顯示邊界線選項設定
\ifthenelse{\equal{\thesisShowBoardLine}{Y}}{
    \usepackage{showframe}
}{}

% --------------------------------------------------
% 前頁內容設定
% --------------------------------------------------

% 製作封面指令
\newcommand{\thesisMakeCoverPage}{
    \begin{titlepage}
        \begin{singlespace}
            \begin{center}
                % 校徽
                \includegraphics[scale=0.1]{\thesisSealImage}\\
                \vspace{10mm}

                % 校系學位
                \fontsize{26pt}{0pt}{\thesisSchoolNameZHTW}\\
                \vspace{5mm}
                \fontsize{26pt}{0pt}{\thesisDeptNameZHTW 系}\\
                \vspace{5mm}
                \fontsize{24pt}{0pt}{\thesisDegreeZHTW 論文}\\
                \vspace{30mm}

                % 題目
                \fontsize{20pt}{0pt}{\thesisTitleZHTW}\\
                \vspace{5mm}
                \fontsize{20pt}{0pt}{\thesisTitleEN}\\
                \vspace{25mm}

                % 使用 @{} 減少間隔空白
                \begin{tabular}{r@{}l}
                    % 研究生列
                    \fontsize{18pt}{0pt}{\makebox[6em][s]{研\hspace{\fill}究\hspace{\fill}生}:}                &
                    \fontsize{18pt}{0pt}{\thesisAuthorNameZHTW}                                               \\[5mm]

                    % 指導教授列
                    \fontsize{18pt}{0pt}{\makebox[6em][s]{指\hspace{\fill}導\hspace{\fill}教\hspace{\fill}授}:} &
                    \fontsize{18pt}{0pt}{\thesisAdvisorNameZHTW~博士}                                           \\[5mm]

                    % 共同指導教授列
                    \ifthenelse{\equal{\thesisHasCoAdvisor}{Y}}{
                    \phantom{共同指導教授:}                                                                       &
                    \fontsize{18pt}{0pt}{\thesisCoAdvisorNameZHTW~博士}                                         \\
                    }{}
                \end{tabular}

                \vfill
                \fontsize{20pt}{0pt}{\thesisDateZHTW}
            \end{center}
        \end{singlespace}
    \end{titlepage}
    \clearpage % 清除頁碼
}

% 製作書名頁指令
\newcommand{\thesisMakeTitlePage}{
    \begin{titlepage}
        \begin{singlespace}
            \begin{center}
                \vspace*{5mm}

                % 中英標題
                \fontsize{16pt}{0pt}{\thesisTitleZHTW}\\
                \vspace{5mm}
                \fontsize{14pt}{0pt}{\thesisTitleEN}\\
                \vspace{10mm}

                % 使用 @{} 減少間隔空白
                \begin{tabular}{l@{}l}
                    % 研究生列
                    \fontsize{13pt}{0pt}{\makebox[5em][s]{研\hspace{\fill}究\hspace{\fill}生}:}                  &
                    \fontsize{13pt}{0pt}{\thesisAuthorNameZHTW}                                                 \\

                    % 指導教授列
                    \fontsize{13pt}{0pt}{{\makebox[5em][s]{指\hspace{\fill}導\hspace{\fill}教\hspace{\fill}授}:}} &
                    \fontsize{13pt}{0pt}{\thesisAdvisorNameZHTW~博士}                                             \\

                    % 共同指導教授列
                    \ifthenelse{\equal{\thesisHasCoAdvisor}{Y}}{
                    \fontsize{13pt}{0pt}{\makebox[5em][s]{}}                                                  &
                    \fontsize{13pt}{0pt}{\thesisCoAdvisorNameZHTW~博士}                                           \\
                    }{}
                \end{tabular}

                \vspace{10mm}

                % 校系論文
                \fontsize{14pt}{0pt}{
                    \thesisSchoolNameZHTW\\
                    \vspace{5mm}
                    \thesisClassNameZHTW\\
                    \vspace{5mm}
                    \thesisDegreeZHTW 論文\\
                }
                \vspace{10mm}

                % 學位聲明
                \fontsize{14pt}{0pt}{
                    A Thesis\\
                    Presented to\\
                    Department of \thesisDeptNameEN\\
                    \thesisSchoolNameEN\\
                    in Partial Fulfillment of the Requirements\\
                    for the Degree of\\
                    \thesisConferredDegree\\
                    in\\
                    \thesisDeptNameEN\\
                }
                \vspace{10mm}

                % 日期校址
                \fontsize{14pt}{0pt}{
                    \thesisDateEN\\
                    \thesisSchoolLocationEN\\
                }

                \vfill
                \fontsize{14pt}{0pt}{\thesisDateZHTW}
            \end{center}
        \end{singlespace}
    \end{titlepage}
    \clearpage % 清除頁碼
}

% 中文摘要環境
\newenvironment{thesisAbstractEnvZHTW}{
    \phantomsection % 目錄超連結錨點
    \addcontentsline{toc}{chapter}{\thesisAbstractTitleZHTW}  % 加入目錄    

    \chapter*{\thesisTitleZHTW}
    \vspace{10mm}

    % 指導教授與研究生
    \begin{center}
        \begin{minipage}[t]{0.5\textwidth}
            \begin{flushleft}
                {\Large 研究生:\thesisAuthorNameZHTW}
            \end{flushleft}
        \end{minipage}% 要加這個不然不會並排
        \begin{minipage}[t]{0.5\textwidth}
            \begin{flushright}
                \def\arraystretch{1} % 設定列高為 1 倍讓兩邊對齊
                \begin{tabular}[t]{l@{}l}
                    {\Large 指導教授:} & {\Large \thesisAdvisorNameZHTW~博士}
                    \ifthenelse{\equal{\thesisHasCoAdvisor}{Y}}{
                    \\[5mm]
                                   & {\Large \thesisCoAdvisorNameZHTW~博士} \\
                    }{}
                \end{tabular}
            \end{flushright}
        \end{minipage}
    \end{center}

    \vspace{5mm}

    % 校與班名
    \begin{center}
        {\Large \thesisSchoolNameZHTW \thesisClassNameZHTW}\\

        \vspace{10mm}

        {\Large \thesisAbstractTitleZHTW}\\

        \vspace{10mm}
    \end{center}

    \par
}{
    % 需先分段 vspace 才能生效
    \par\vspace{10mm}

    % 印出中文關鍵字
    % \noindent 取消因 \par 產生的縮排
    % 加 % 消除 foreach 後空格
    \noindent{\thesisKeywordTagZHTW
        \foreach \keyword [count=\i] in \thesisKeywordsZHTW {%
            \ifnum\i=1% 
                \keyword%
            \else%
                \thesisKeywordSymbolZHTW\keyword%
            \fi%
        }
    }

    \pagebreak % 需改
}

% 英文摘要環境
\newenvironment{thesisAbstractEnvEN}{
    \phantomsection % 目錄超連結錨點
    \addcontentsline{toc}{chapter}{\thesisAbstractTitleEN} % 加入目錄

    \chapter*{\thesisTitleEN}
    \vspace{10mm}

    % 指導教授與研究生
    \begin{center}
        \begin{minipage}[t]{0.4\textwidth}
            \begin{flushleft}
                {\large Student:~\thesisAuthorNameEN}
            \end{flushleft}
        \end{minipage}% 要加這個不然不會並排
        \begin{minipage}[t]{0.6\textwidth}
            \begin{flushright}
                \def\arraystretch{1} % 設定列高為 1 倍讓兩邊對齊
                \begin{tabular}[t]{l@{}l}
                    {\large Advisor:} & {\large ~Dr.~\thesisAdvisorNameEN}
                    \ifthenelse{\equal{\thesisHasCoAdvisor}{Y}}{
                    \\[5mm]
                                      & {\large ~Dr.~\thesisCoAdvisorNameEN} \\
                    }{}
                \end{tabular}
            \end{flushright}
        \end{minipage}
    \end{center}

    \vspace{5mm}

    % 校與班名
    \begin{center}
        {\large \thesisClassNameEN\\
            \thesisSchoolNameEN}\\
        \vspace{10mm}
        {\Large \thesisAbstractTitleEN}\\
        \vspace{10mm}
    \end{center}

    \par
}{
    % 需先分段 vspace 才能生效
    \par\vspace{10mm}

    % 印出英文關鍵字
    % \noindent 取消因 \par 產生的縮排
    % 加 % 消除 foreach 後空格
    \noindent{\thesisKeywordTagEN
        \foreach \keyword [count=\i] in \thesisKeywordsEN {%
            \ifnum\i=1%
                \keyword%
            \else%
                \thesisKeywordSymbolEN~\keyword%
            \fi%
        }
    }
    \pagebreak % 需改
}

% 致謝指令
\newcommand{\thesisAcknowledgement}{
    \chapter*{\thesisAcknowledgementTitle}

    \phantomsection % 目錄超連結錨點
    \addcontentsline{toc}{chapter}{\thesisAcknowledgementTitle} % 加入目錄
}

% 符號說明表環境
\newenvironment{thesisDenotationEnv}[1][2.5cm]{
    \chapter*{\thesisDenotation}

    \phantomsection % 目錄超連結錨點
    \addcontentsline{toc}{chapter}{\thesisDenotation} % 加入目錄

    \begin{list}{}{
            \renewcommand\makelabel[1]{##1\hfill}
            \setlength{\labelwidth}{#1}                     % 符號欄寬度
            \setlength{\labelsep}{0.5cm}                    % 標籤與列表文字距離
            \setlength{\itemindent}{0cm}                    % 標籤縮進
            \setlength{\leftmargin}{\labelwidth+\labelsep}  % 標籤左邊界
            \setlength{\rightmargin}{0cm}                   % 標籤右邊界
            \setlength{\parsep}{0cm}                        % 段落間距
            \setlength{\itemsep}{18pt}                      % 標籤間距
            \setlength{\listparindent}{0cm}                 % 段落縮排
            \setlength{\topsep}{0pt}                        % 標籤與上文距離
        }
        }{
    \end{list}
    \pagebreak % 需改
}

% --------------------------------------------------
% 目錄設定
% --------------------------------------------------
% 載入套件 titletoc 設定目錄格式
% \titlecontents{標題名稱}[左間距]
% {標題格式}
% {標題內容}
% {無序號標題}
% {指引線與頁碼}
% [下間距]
% --------------------------------------------------

% 設定目錄樣式

\setcounter{tocdepth}{2} % 設定目錄深度至 2 => 子節
\setcounter{secnumdepth}{2} % 設定編號深度 2 => 子節

\renewcommand{\contentsname}{\centerline{\thesisTableOfContents}} % 設定目錄標題

\titlecontents{chapter}[0em]
{}
{\thesisChapterTitleToc} % 須改中英間距於 config
{}
{\titlerule*[0.5pc]{.}\contentspage}
{}

% 設定節目錄樣式 (保留)

% 設定子節目錄樣式 (保留)

% 設定表目錄樣式
\renewcommand{\listtablename}{\centerline{\thesisListOfTables}} % 設定表目錄頁標題

\renewcommand{\tablename}{\thesisTableName} % 設定表目錄頁標號,如 表 1, 表 2 .... 

\titlecontents{table}[0em]
{\tablename~\thecontentslabel~}
{}
{}
{\titlerule*[0.5pc]{.}\contentspage}
{}

% 設定圖目錄樣式
\renewcommand{\listfigurename}{\centerline{\thesisListOfFigures}} % 設定圖目錄頁標題
\renewcommand{\figurename}{\thesisFigureName} % 設定圖目錄頁標號,如 圖 1, 圖 2 ....

\titlecontents{figure}[0em]
{\figurename~\thecontentslabel~}
{}
{}
{\titlerule*[0.5pc]{.}\contentspage}
{}

% 製作目錄指令
\newcommand{\thesisMakeTableOfContents}{
    \clearpage % 需改

    % 加入目錄與目錄超連結
    \phantomsection
    \addcontentsline{toc}{chapter}{\thesisTableOfContents}

    \tableofcontents
    \newpage
}

% 製作表目錄指令
\newcommand{\thesisMakeListOfTables}{
    \clearpage % 需改

    % 加入目錄與目錄超連結
    \phantomsection
    \addcontentsline{toc}{chapter}{\thesisListOfTables}

    \listoftables
    \newpage % 需改
}

% 製作圖目錄指令
\newcommand{\thesisMakeListOfFigures}{
    \clearpage % 需改

    % 加入目錄與目錄超連結
    \phantomsection
    \addcontentsline{toc}{chapter}{\thesisListOfFigures}

    \listoffigures

    \newpage % 需改
}

% --------------------------------------------------
% 標題設定
% --------------------------------------------------
% 載入套件 titlesec 設定各層標題樣式
% \titleformat{command}[shape]{format}{label}{sep}{before}[after]
% - command: 代表欲重新定義之標題階層,包含 \part, \chapter, \section, \paragraph
% - shape: 設定段落形狀,可選參數有 hang, block, display
% - format: 定義標題外觀,如:位置、字體大小粗細…
% - label: 定義標籤前綴標號
% - sep: 定義標號與標題內容間隔距離
% - before: 標題內容前增加文字
% - after: 標題內容後增加文字
%
% 僅設定標題樣式
% \titleformat*{command}{format}
%
% 強制設定標題與內容間距( \titlespacing 非嚴格)
% \titlespacing*{command}{left}{before-sep}{after-sep}[right]
% - command: 標題指令
% - left: 標題左邊的缩排
% - before-sep: 標題上方的空白間距
% - after-sep: 標題下方的空白間距
% - right(可選): 標題右邊的缩排(一般很少用)。
% --------------------------------------------------

% 設定章標題樣式
\titleformat{\chapter}
{\centering\LARGE}
{\thesisChapterTitle}
{1em} % 或許需改中英間隔大小
{}

\titlespacing*{\chapter}{0pt}{0pt}{12pt} % 設定章標題下方留空一行

% 設定節標題樣式
\titleformat*{\section}{\Large}
\titlespacing*{\section}{0pt}{12pt}{12pt} % 設定節標題上下留空

% 設定子節標題樣式
\titleformat*{\subsection}{\large}
\titlespacing*{\subsection}{0pt}{12pt}{12pt} % 設定子節標題上下留空

% 表標題設定
\counterwithout{table}{chapter} % 不按章節排序

\captionsetup[table]{ % caption
    justification=centering, % 標題置中
    position=top, % 標題在表上方
    font=normalsize, % 字體大小為 10pt
    labelsep=space, % 標號後使用空白間隔
    skip=10pt, % 標題與表的間距 10pt
}

% 圖標題設定
\counterwithout{figure}{chapter} % 不按章節排序

\captionsetup[figure]{ % caption
    justification=centering, % 標題置中
    position=bottom, % 放在圖下方
    font=normalsize, % 字體大小為 10pt
    labelsep=space, % 標號後使用空白間隔
    skip=10pt, % 標題與圖的間距 10pt
}

% 子圖標題設定(保留)

% 公式標號設定
\counterwithout{equation}{chapter} % 不按章節排序

% --------------------------------------------------
% 頁碼設定
% --------------------------------------------------

\newcommand{\thesisFrontStart}{
    \pagenumbering{roman} % 設定羅馬字頁碼
    \setcounter{page}{1}  % 中文摘要為前頁第一頁
}

\newcommand{\thesisContentStart}{
    \pagenumbering{arabic} % 重設為數字頁碼
    \setcounter{page}{1}   % 重設頁碼計數
}

% --------------------------------------------------
% 內文設定
% --------------------------------------------------

% 設定中文行距為 1,英文行距為 1.5
\ifthenelse{\equal{\thesisUseZHTW}{Y}}{
    \linespread{1}
}{
    \linespread{1.5}
}

% 設定段落首行縮排
\setlength{\parindent}{2em}

% 設定段落間距
\setlength{\parskip}{6pt}

% 設定根據中文規則斷詞
\XeTeXlinebreaklocale "zh"
\XeTeXlinebreakskip = 0pt plus 1pt

% 設定懲罰係數
\doublehyphendemerits=10000 % 不允許連續兩行使用連字號
\brokenpenalty=10000 % 單字不跨欄/跨頁斷開
\widowpenalty=9999 % 幾乎不允許頁面底部僅有一行段落
\clubpenalty=9999 % 幾乎不允許頁面頂部僅有一行段落
\interfootnotelinepenalty=9999 % 幾乎不允許註腳在行內斷開

% --------------------------------------------------
% 圖表設定
% --------------------------------------------------

% graphicx 根目錄設定
\graphicspath{{\thesisFiguresPath}}

% pgfplots 初始設定
\usepgfplotslibrary{colorbrewer}
\pgfplotsset{compat=newest,compat/show suggested version=false}

% 製作浮水印指令
\newcommand{\thesisMakeWatermark}{
    \ifthenelse{\equal{\thesisUseWatermark}{Y}}{
        \backgroundsetup{
            contents=\includegraphics{\thesisWatermarkImage},
            firstpage=false,
            angle=0,
            opacity=0.5,
            hshift=0mm,
            vshift=0mm,
            scale=0.1, % 大小
        }
    }{}
}

% --------------------------------------------------
% 數學、演算法設定
% --------------------------------------------------

% 演算法區塊
\definecolor{codegreen}{rgb}{0,0.6,0}
\definecolor{codegray}{rgb}{0.5,0.5,0.5}
\definecolor{codepurple}{rgb}{0.58,0,0.82}
\definecolor{backcolour}{rgb}{0.95,0.95,0.92}

\lstdefinestyle{codestyle}{
    backgroundcolor=\color{backcolour},
    commentstyle=\color{codegreen},
    keywordstyle=\color{magenta},
    numberstyle=\tiny\color{codegray},
    stringstyle=\color{codepurple},
    basicstyle=\textrm\footnotesize\ttfamily,
    breakatwhitespace=false,
    breaklines=true,
    captionpos=b,
    keepspaces=true,
    numbers=left,
    numbersep=5pt,
    showspaces=false,
    showstringspaces=false,
    showtabs=false,
    tabsize=2
}

\lstset{style=codestyle}

% --------------------------------------------------
% PDF、超連結設定
% --------------------------------------------------

% 超連結
\hypersetup{
    colorlinks=true,
    linkcolor=black,
    filecolor=gray,
    anchorcolor=blue,
    citecolor=blue
}

% --------------------------------------------------
% 參考文獻設定
% --------------------------------------------------

% 更換參考文獻標題
\renewcommand{\bibname}{\thesisBibliographyTitle}

% biblatex 加入參考文獻檔案位置
\addbibresource{\thesisRefBIB}

% biblatex 修改英文參考文獻顯示符號

% \DefineBibliographyStrings{english}{
%     andothers = {et al.},
%     editor = {ed.},
%     editors = {eds.}
% }

% biblatex 修改中文參考文獻顯示符號

% \DefineBibliographyStrings{chinese}{
%     andothers = {等},
%     editor = {編},
%     editors = {編}
% }

% 製作參考文獻指令
\newcommand{\thesisMakeReference}{
    \clearpage

    \chapter*{\thesisBibliographyTitle}

    \phantomsection % 目錄超連結錨點
    \addcontentsline{toc}{chapter}{\thesisBibliographyTitle} % 加入目錄

    % 插入參考文獻
    \ifthenelse{\equal{\thesisSplitBIBByLang}{Y}}{
        \section*{中文部分}
        \printbibliography[
            keyword={chinese},
            heading=none,
        ]

        \section*{外文部分}
        \printbibliography[
            notkeyword={chinese},
            heading=none,
        ]
    }{
        \printbibliography[
            heading = none
        ]
    }
}

% --------------------------------------------------
% 附錄設定
% --------------------------------------------------

% 附錄指令
\newcommand{\thesisAppendix}[2]{
    \ifthenelse{\equal{\thesisUseZHTW}{Y}}{
        \def\thesisCurrAppendixTitle{
            \thesisAppendixTitle{#1}、#2
        }
    }{
        \def\thesisCurrAppendixTitle{
            \thesisAppendixTitle{#1}~#2
        }
    }

    \chapter*{\thesisCurrAppendixTitle}

    \phantomsection % 目錄超連結錨點
    \addcontentsline{toc}{chapter}{ \thesisCurrAppendixTitle } % 加入目錄
}
